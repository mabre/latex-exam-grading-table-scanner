% \begin{tabularx}{\textwidth}{X}
% \hhline{-}\\
% 
% \end{tabularx}

\hrule


\vspace{\baselineskip}

\begin{center}
    \ifnum \numpoints < 100
        \htword{$\Sigma$}
    \else
        \htword{\hspace{0.5mm}Summe\hspace{0.5mm}} % Sorgt für breiteres Summenfeld
    \fi
    \cellwidth{.8cm}
    
    \usetikzlibrary{calc}
    \hsword{
        Achieved
        \begin{tikzpicture}[overlay]
            \def\base{
                -\numquestions*\baselineskip*3*.5-\baselineskip*4.53*.5+\baselineskip*4+44.2,0.51
            }
            \coordinate (first_dashed) at ($(\base)+(3*1.236/3+0*1.236-2.572,-.7)$);
            \draw[gray, dashed] (first_dashed) -- ++(0,-.73); % first dashed line
            \foreach \x in {1,...,\numquestions} {
                \foreach \y in {1,2} {
                    \draw[gray, dashed] ($(\base)+(\y*1.236/3+\x*1.236-2.572,0)$) -- ++(0,-.73);
                    \draw[gray, dashed] ($(\base)+(\y*1.236/3+\x*1.236-2.572,-.7)$) -- ++(0,-.73);
                }
                \draw[gray, dashed, line width=0.35mm] ($(\base)+(2*1.236/3+\x*1.236-2.572,-.62)$) -- ++(0,-.13);
                \draw[gray, dashed] ($(\base)+(3*1.236/3+\x*1.236-2.572,-.7)$) -- ++(0,-.73);
            }
            \ifnum \numpoints < 100
            \coordinate (last_dashed) at ($(\base)+(3*1.236/3+\numquestions*1.236-1.342,-.7)$);
            \draw[gray, dashed] ($(\base)+(1*1.336/3+\numquestions*1.236-1.372,0)$) -- ++(0,-.73);
            \draw[gray, dashed] ($(\base)+(1*1.336/3+\numquestions*1.236-1.372,-.7)$) -- ++(0,-.73);
            \draw[gray, dashed] ($(\base)+(2*1.336/3+\numquestions*1.236-1.372,0)$) -- ++(0,-.73);
            \draw[gray, dashed] ($(\base)+(2*1.336/3+\numquestions*1.236-1.372,-.7)$) -- ++(0,-.73);
            \draw[gray, dashed, line width=0.35mm] ($(\base)+(2*1.336/3+\numquestions*1.236-1.372,-.62)$) -- ++(0,-.13);
            \draw[gray, dashed] (last_dashed) -- ++(0,-.73); % last dashed line
            \else
            \coordinate (last_dashed) at ($(\base)+(3*1.336/3+\numquestions*1*1.236-1.372,-.7)$);
            \draw[gray, dashed] ($(\base)+(1*1.336/3+\numquestions*1.236-1.372,0)$) -- ++(0,-.73);
            \draw[gray, dashed] ($(\base)+(1*1.336/3+\numquestions*1.236-1.372,-.7)$) -- ++(0,-.73);
            \draw[gray, dashed] ($(\base)+(2*1.336/3+\numquestions*1*1.236-1.372,0)$) -- ++(0,-.73);
            \draw[gray, dashed] ($(\base)+(2*1.336/3+\numquestions*1*1.236-1.372,-.7)$) -- ++(0,-.73);
            \draw[gray, dashed] ($(\base)+(3*1.336/3+\numquestions*1*1.236-1.372,0)$) -- ++(0,-.73);
            \draw[gray, dashed] (last_dashed) -- ++(0,-.73); % last dashed line
            \fi
            % eine Zellbreite ist 0,44
            \node at ($(first_dashed) + (-0.7, -0.48)$) {\includegraphics{aruco_bottom_left.png}}; % untere rechte Ecke muss bündig mit Tabelle sein, Abstand eine Zellbreite
            \node at ($(last_dashed) + (0.7, 1.91)$) {\includegraphics{aruco_top_right.png}}; % obere linke Ecke muss bündigt mit Tabelle sein, Abstand eine Zellbreite
            \node at ($(last_dashed) + (0.7, -0.48)$) {\includegraphics{aruco_bottom_right.png}}; % untere linke Ecke muss bündigt mit Tabelle sein, Abstand eine Zellbreite
            \node at ($(first_dashed) + (-0.7, 2.42)$) {\includegraphics{aruco_top_left.png}}; % untere rechte Ecke muss bündig mit Tabelle sein, Abstand eine Zellbreite
        \end{tikzpicture}
    }
    \setcounter{numquestions}{0} % wg. Serienbrief
    % NB: Hier gibts ein »Dimension too large.«, falls im vorherigen Compilerdurchgang numquestions exporbitant hoch geworden ist (das passiert, falls man dummerweise die Lösung als Serienbrief erstellt, dann wir der Reset in der vorherigen Zeile nämlich nicht ausgeführt; mit der aktuellen Makrodefinition sollte das aber nicht mehr passieren).
    \partialgradetable{myrange}[h][questions]
    \setcounter{numpoints}{0} % nicht sicher, das das ein kluger Hack ist; benötigt für saubere Breite in Summenspalte bei Serienbrief
\end{center}
